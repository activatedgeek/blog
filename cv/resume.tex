\documentclass{resume}

\usepackage[left=0.4 in,top=0.4in,right=0.4 in,bottom=0.4in]{geometry}
\usepackage{enumitem}

\newcommand{\tab}[1]{\hspace{.2667\textwidth}\rlap{#1}}
\newcommand{\itab}[1]{\hspace{0em}\rlap{#1}}

\name{Sanyam Kapoor}
\address{+1 (201) 710-0784 \\ sanyam@nyu.edu \\ www.sanyamkapoor.com}

\begin{document}

\begin{rSection}{Education}

{\bf Courant Institute of Mathematical Sciences, New York University} \hfill Sept 2017 - May 2019 \\
\it{Masters in Computer Science}

\vspace{-0.4em}

{\bf Indian Institute of Technology (IIT) Hyderabad, India} \hfill Aug 2012 - May 2016 \\
\it{Bachelors in Computer Science and Engineering}
\begin{list}{$\cdot$}{\leftmargin=2em}
\itemsep -0.7em \vspace{-0.5em}
\item {\bf TODAI Award, University of Tokyo, Japan} for outstanding academic performance, 2013
\item {\bf Academic Excellence Award} for highest Semester GPA, 2012
\end{list}

\end{rSection}

\begin{rSection}{Research Experience}

\begin{rSubsection}{Food Volume Estimation for Dietary Purposes}{Jul 2017 - present}{Advisor: Prof. Dennis Shasha}{}
\item Investigating techniques for 3D Geometry Reconstruction from a given set of stereo images of food on plate
\item Using ideas for Point Cloud Generation based on Depth Fusion Mapping

\end{rSubsection}

\begin{rSubsection}{Sports Video Recognition via Dense Trajectory Features}{Sept 2015 - Nov 2015}{Advisor:  Prof. Vineeth N Balasubramanian}{}
\item Utilized HOG, HOF and MBH detectors for voxel features and tracked the Optical Flow via DenseTrack
\item Trained an SVM on 8 sports from UCF Sports Dataset with 30\% accuracy

\end{rSubsection}

\begin{rSubsection}{eDrishti, Engagement Level Detection in MOOC Videos}{Jan 2015 - Apr 2015}{Advisor:  Prof. Vineeth N Balasubramanian}{}
\item Generated facial features for expression using Gabor Filters on a self-curated dataset of 200 videos
\item Recognized engagement levels (low, medium, high) with an accuracy of 67\%

\end{rSubsection}

\end{rSection}

\begin{rSection}{Professional Experience}

\begin{rSubsection}{Software Engineer, Headout \it{Bengaluru, India} (\bf{headout.com})}{Dec 2016 - Jul 2017}{}{}
\item Designed and implemented experiments for the Growth Team, built client APIs for the Platform team
\item Led internal developer tooling, reduced developer on-boarding from a full day to half an hour
\item Led migration to a CI/CD infrastructure for automated deployments based on Docker and AWS
\item Slashed application rollback downtime by 100\%

\end{rSubsection}

\begin{rSubsection}{Co-Founder, StoryXpress, \it{Hyderabad, India} (\bf{storyxpress.co})}{May 2013 - Aug 2016}{}{}
\item Founded the Cloud Video Service for large scale video creation from static content
\item Built an in-house Video Rendering Engine on top of OpenGL, generated around 2000 videos per month
\item Led development of enterprise APIs and Web Application for enterprises like Target and TradeIndia

\end{rSubsection}

\end{rSection}

\begin{rSection}{Honors and Awards}
  \begin{itemize}[label={},topsep=0pt,itemsep=-0.5ex,partopsep=1ex,parsep=1ex,leftmargin=0.25em]
  \item  {\bf StackOverflow Top Contributor}, top 0.6\% overall among 7+ million members, 2017
  \item {\bf NASSCOM Emerge 50}, {\it StoryXpress} among 500+ startups across India for innovation impact, 2015
  \item {\bf HYSEA Best Software Product, Student Innovation}, {\it StoryXpress} among 100+ startups, 2015
  \item {\bf Microsoft Build the Shield, India}, First Runner Up among 280 teams, 2015
  \item {\bf ACM ICPC Amritapuri Regionals} finalist among 1500+ teams, 2013
  \end{itemize}
\end{rSection}

\begin{rSection}{Relevant Projects}
  \begin{itemize}[label={},topsep=0pt,itemsep=-0.5ex,partopsep=1ex,parsep=1ex,leftmargin=0.25em]
  \item {\bf MariaDB Scheduler} - A Proof-of-Concept on top of DC/OS based on a very early-stage framework
  \item  {\bf Docker Consul} - A Docker container with networking tweaks. Used for quorum management. 25000+ pulls.
  \item {\bf QuickSlots v2.0} - A Timetable Scheduler modeled as Min-Cost Bipartite Matching Problem
  \item {\bf COOL Compiler} - Lexing, Parsing and Semantic Phases for the Classroom Object Oriented Language
  \end{itemize}
\end{rSection}

\begin{rSection}{Technical Skills}

  \begin{itemize}[label={},topsep=0pt,itemsep=-0.5ex,partopsep=1ex,parsep=1ex,leftmargin=0.25em]
  \item  {\bf Programming Languages}: Python, Go, C, C++, Node, Java
  \item {\bf Technologies}: OpenCV, Scikit-learn, OpenGL, Git, MySQL, React, Docker, Ansible, Vagrant, Redis
  \end{itemize}

\end{rSection}

\end{document}
