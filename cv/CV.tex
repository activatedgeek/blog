\documentclass[margin,line]{res}

\usepackage{verbatim}
\usepackage{anysize}

\oddsidemargin -.5in
\evensidemargin -.5in
\marginsize{0.5in}{0.5in}{1in}{1in}

\textwidth=6.0in

\itemsep=0in
\parsep=0in

\newenvironment{list1}{
  \begin{list}{\ding{113}}{%
    \setlength{\itemsep}{0in}
    \setlength{\parsep}{0in} \setlength{\parskip}{0in}
    \setlength{\topsep}{0in} \setlength{\partopsep}{0in}
    \setlength{\leftmargin}{0.17in}}}{\end{list}}
\newenvironment{list2}{
  \begin{list}{$\bullet$}{%
    \setlength{\itemsep}{0in}
    \setlength{\parsep}{0in} \setlength{\parskip}{0in}
    \setlength{\topsep}{0in} \setlength{\partopsep}{0in}
    \setlength{\leftmargin}{0.2in}}}{\end{list}}

\begin{document}

\name{Sanyam Kapoor \vspace*{.05in}}

\begin{resume}

\section{\sc Contact Information}

  \vspace{.05in}

  \begin{tabular}{@{}p{2.9in}p{6in}}
	{\it Website:} www.sanyamkapoor.com & {\it Email:} cs12b1043@iith.ac.in \\
	{\it Github:} github.com/activatedgeek & {\it Mobile:} +91 83107 12634  \\
  \end{tabular}

\section{\sc Research Interests}
  Computer Vision and Perception, Object Recognition, Deep Learning

\section{\sc Education}

  {\bf Indian Institute of Technology, Hyderabad}, India \hfill 2012 - 2016 \\
  	B. Tech. Computer Science and Engineering,
	GPA: 8.62 / 10 ({\bf Top 10})

  \vspace*{-2.5mm}

  {\bf La Montessori School}, Kullu, HP, India \hfill 2010 - 2012 \\
	Intermediate (Physics, Mathematics, Chemistry),
    Percentage Score: 93.4\% ({\bf Rank 1})

  \vspace*{-2.5mm}

  {\bf La Montessori School}, Kullu, HP, India \hfill 2010 \\
	Matriculation,
    GPA: 9.8 / 10, ({\bf Top 2})

\section{\sc Honors and Awards}

  {\bf StackOverflow Top Contributor}, top 0.6\% overall among 7 million members

  \vspace*{-2.5mm}

  {\bf NASSCOM Emerge 50}, {\it StoryXpress} among over 500 startups across India for innovation impact and growth, 2015

  \vspace*{-2.5mm}

  {\bf HYSEA Best Software Product, Student Innovation}, {\it StoryXpress} among over 100 startups in Hyderabad, 2015

  \vspace*{-2.5mm}

  {\bf Microsoft Build the Shield, India}, First Runner Up among 280 teams, 2015

  \vspace*{-2.5mm}

  {\bf Best Science \& Technology Club}, Core Member, {\it Infero}, IIT Hyderabad for contributions to student community, 2014

  \vspace*{-2.5mm}

  {\bf ACM ICPC Amritapuri Regionals} finalist among over 1500 teams, 2013

  \vspace*{-2.5mm}

  {\bf TODAI Award}, {\it University of Tokyo, Japan} for outstanding academic performance ({\bf Rank 1}), cash prize of USD 1500, 2013

  \vspace*{-2.5mm}

  {\bf Academic Excellence Award}, {\it IIT Hyderabad} for highest GPA, 2012

  \vspace*{-2.5mm}

  {\bf Top 0.5\%}, {\it IIT JEE} (0.5 million students) and {\bf Top 0.1\%}, {\it AIEEE} (1.2 million students), 2012

\section{\sc Research Experience}

  {\bf Sports Recognition via Dense Trajectory Features} \hfill September - November, 2015 \\
  	{\em Advisor: Dr. Vineeth N Balasubramanian}, {\em Technologies: Scikit-Learn} \vspace{0.15 \baselineskip} \\
    I investigated feature extraction in videos using {\it HOG}, {\it HOF} and {\it MBH} detectors (based on optical flow analysis with {\it DenseTrack}) and represented these as Bag of Visual Words. It was trained via an {\it SVM (Chi-Squared Kernel)} on 8 sports from {\it UCF Sports Dataset} and achieved accuracy of 30\%.

  \vspace*{-2.5mm}

  {\bf eDrishti, Engagement Level Detection in Videos} \hfill January - April, 2015 \\
  	{\em Advisor: Dr. Vineeth N Balasubramanian}, {\em Technologies: Scikit-Learn, Matlab, Weka} \vspace{0.15 \baselineskip} \\
  	I built a One-Vs-Rest SVM Classifier to recognize engagement levels (high, medium or low) of viewers during a MOOC video. It involved extracting fine facial features using {\it Gabor Filters}, achieving an accuracy of 67\% on a self-curated dataset of over 200 video clips.

  \vspace*{-2.5mm}

  {\bf Partial Face Detection} \hfill August - November, 2014 \\
  	{\em Advisor: Dr. Vineeth N Balasubramanian}, {\em Technologies: Scikit-Learn} \vspace{0.15 \baselineskip} \\
  I investigated state-of-the-art methods to detect partial faces in images with more natural environments. Occlusions, illumination variations and shadows make this problem challenging. {\it Graph Based Image Segmentation and Coloring} was used for spatial similarity analysis.

\section{\sc Professional Experience}

  {\bf Software Engineer, Headout}, Bengaluru, India, {\it headout.com}  \hfill December 2016 - \\
    I am part of the {\it Platform} team which handles development of APIs and client presentation layers. I also successfully led charge to move all services to a CI/CD enabled deployment pattern and actively engaged in internal developer tooling on top of Docker.

  \vspace*{-2.5mm}

  {\bf Author, MariaDB Service}, {\it github.com/dcos-labs/dcos-mariadb-service} \hfill September 2016 - \\
    I am building an Open Source distributed scheduler for {\it MariaDB} on top of {\em Mesos}. It enables automated scheduling, scaling and maintenance of large-scale SQL databases deployed on {\it Datacenter Operating System (DC/OS)}. I've contributed container support to the {\it Mesos Framework Java SDK}.

  \vspace*{-2.5mm}

  {\bf Co-Founder, StoryXpress}, Hyderabad, India, {\it storyxpress.co}  \hfill May 2013 - August 2016 \\
    I co-founded a cloud service for large scale automated video content creation. I led complete engineering efforts from development to deployment. This included building the core {\it Video Rendering Engine} on top of {\it OpenGL} from scratch and the enterprise API used by large organizations like {\it Target} and {\it TradeIndia}. I also handled complete deployment automation on top of {\it AWS}, {\it Azure} and {\it Digital Ocean}.

\section{\sc Selected Projects}

  {\bf Docker Consul} \hfill April 2016 \\
	I built an Open Source {\it Docker} container image to deploy a quorum management technology called {\it Consul}. It included some networking tweaks to be readily deployed to data centers. It has close to {\em 25000 pulls} since its creation.

  \vspace*{-2.5mm}

  {\bf QuickSlots v2.0} \hfill February - April, 2016 \\
  	{\it Advisor: Dr. Ramakrishna Upadrasta, Dr. MV Panduranga Rao} \vspace{0.15 \baselineskip} \\
    I built a Timetable Scheduler for the {\it Computer Science Department, IIT Hyderabad}. The input to the system is faculty's preferences and the output is a consolidated time-table. The problem was modelled as {\em Min Cost Bipartite Matching}. A mathematical formulation for edge weights was created based on factors like time and venue conflicts, student batch size and course kind.

  \vspace*{-2.5mm}

  {\bf COOL Compiler} \hfill September - November, 2014 \\
  	{\it Advisor: Dr. Ramakrishna Upadrasta} \vspace{0.15 \baselineskip} \\
  	I built a compiler for the {\it Classroom Object Oriented Language}. This included understanding and implementing the {\em Lexing}, {\em Parsing} and {\em Semantic} phases for the {\it COOL} grammar. The implementation was done in C++ and including machine code generation for the {\it MIPS} architecture.

  \vspace*{-2.5mm}

  {\bf DownloadPlus} \hfill October - November 2014 \\
  	I built a Multi-threaded Download Manager in Java. It exploited the chunked downloading feature prescribed by the {\it HTTP/1.1 specification (RFC 2616)} to allow parallel downloading multiple byte ranges of a remote file, decreasing download times. It used {\it Java Sockets} and {\it Java Threads} API.

  \vspace*{-2.5mm}

  {\bf GossipBox} \hfill September - October 2014 \\
    I built a Client-Server Chat System in {\it C++} using the {\it POSIX Socket API}. It maintained an in-memory hash table of connected clients which were kept in regular sync and automatic garbage collection once sockets timed out.

\section{\sc Technical Skills}

  {\bf Technologies}: Git, MySQL, Scikit-learn, OpenGL, React, Docker, Redis \\
  {\bf Programming Languages}: Go, Java, C, C++, Python, Node, MATLAB

\section{\sc Relevant Coursework}

  {\bf Core:} Computer Vision, Soft Computing, Numerical Linear Algebra for Data Analysis, Predictive Analytics \& Knowledge Discovery, Operating Systems, Compiler Design, Computer Networks, Data Structure and Algorithms, Discrete Mathematics \\
  {\bf Independent Coursework:} Machine Learning ({\it Coursera}), Neural Networks ({\it Coursera}) \\
  {\bf Breadth:} Real Analysis, Complex Analysis, Calculus, Economics

\begin{comment}
\section{\sc Positions of Responsibility}

  {\bf Volunteer, National Service Scheme}, IIT Hyderabad \hfill 2012 - 2016 \\
    I donated clothes to {\it old-age homes}. I gave {\it lectures on Basic Science and Mathematics} to primary school students. I was actively involved in {\it planting and watering greens} around the campus and was a volunteer for the {\it campus cleanliness drive}.

  \vspace*{-2.5mm}

  {\bf Core Member, Infero}, IIT Hyderabad \hfill 2013 - 2014 \\
  	I was a Core Member of {\em Infero}, the Programming Club of IIT Hyderabad. I prepared and delivered presentations on {\em Introduction to C} and {\em Introduction to Python} to freshmen. My team also introduced {\it Code Nights} to encourage Competitive Programming in the student community. We won the {\em Best Sci-Tech Club Award} for these contributions.

  \vspace*{-2.5mm}

  {\bf Events Manager, Entrepreneurship Cell}, IIT Hyderabad \hfill 2013 - 2014 \\
    I was responsible for managing campus and city-wide events for the club. My team introduced the {\em first ever lecture series} called {\it Spark} which invited successful entrepreneurs for lectures and interaction. I also drafted a {\it Business Plan Competition} as a part of the technical fest.

\section{\sc References}

  \begin{tabular}{@{}p{1.85in}p{2in}p{2in}}

  {\bf Dr. Vineeth N \newline Balasubramanian } & {\bf Dr. Ramakrishna \newline Upadrasta} & {\bf Joerg Schad} \\
        {\it Assistant Professor} & {\it Assistant Professor} & {\it Distributed Systems Engineer} \\
        {\it IIT Hyderabad} & {\it IIT Hyderabad} & {\it Mesosphere Inc} \\
        {\it Hyderabad, India} & {\it Hyderabad, India} & {\it Hamburg, Germany} \\
        {\it vineethnb@iith.ac.in} & {\it ramakrishna@iith.ac.in} & {\it joerg@mesosphere.io} \\

  \end{tabular}
\end{comment}

\end{resume}
\end{document}
